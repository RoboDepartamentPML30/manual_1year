\chapter{Движение змейкой}
{\bfseries Анонс:}\\\\
Зачетное занятие по теме «Движение по окружности. Основы работы в RobotC».\\\\
{\bfseries Цели:}
\begin{itemize}
	\item{}{\bfseries Обучающие:} проконтролировать степень усвоения следующих основных знаний, умений и навыков, изученных и сформированных на предыдущих занятиях.
	\item{}{\bfseries Развивающая:} Способствовать развитию  у школьников самостоятельности мышления в целях развития интеллектуальных способностей; умению переносить знания и навыки в новые ситуации.\\
\end{itemize}	
{\bfseries Ход занятия:}\\\\
\begin{tabular}{lll}
	\hyperlink{lesson13x1}{1. Организационный момент} & Презентация & (5 мин)\\
	\hyperlink{lesson13x2}{2. «Змейка»} & Практикум & (90 мин) \\
	\hyperlink{lesson13x3}{3. Анализ технических решений} & Рефлексия & (20 мин) \\
\end{tabular}\\\\

{\hypertarget{lesson13x1}{\blackBlueText{I. Организационный момент}}}\\\\ 

У каждой команды (2 человека) есть свое рабочее место, на каждые 3 команды приходится 1 испытательный полигон, с расчерченными местами для банок (схема в Приложении). Учащиеся могут пользоваться любой справочной литературой.
\clearpage
{\hypertarget{lesson13x2}{\blackBlueText{II. «Змейка»}}}\\\\

Задача: Проехать змейкой между 4 банками, расставленными вдоль одной линии. Расстояние между краями банок 30 см. Старт с отметки в 30 см от первой банки, финиш на отметке в 30 см от последней банки.\\\\

\greenText{Рис}\\\\

Задание считается выполненным, если робот пересек черту финиша, проехав змейкой, и не коснулся ни одной банки. Время прохождения дистанции не ограничено.

{\slshape Наиболее популярными вариантами решения являются: движение дугами окружности, движение ломанной с поворотами под 90 градусов. Важно, что при последнем варианте задачу можно решить, даже усвоив лишь половину материала (движение прямо, поворот на месте).
	
	По успешному завершению задания рекомендуется как-то отметить это достижение учащихся. Причем интересно оказывается не вручение каких-либо призов или грамот, тем более что задание так или иначе должно быть выполнено всеми, и важен лишь факт выполнения задания, а не скорость. В качестве возможных вариантов автором использовалось вручение значков с символикой кружка; получение персонального логина и пароля для компьютера. Важно подчеркнуть, что дети справились с каким-то важным этапом и стали на ступень выше, вошли в сообщество настоящих робототехников.}\\\\

{\hypertarget{lesson13x3}{\blackBlueText{III. Анализ технических решений}}}\\\\

\begin{enumerate}
	\item Были ли у вас проблемы связанные с инерционностью двигателя? Как вы с ними боролись?
	\item Почему робот мог объехать первые пару банок, но неизбежно все хуже и хуже объезжал последующие?
	\item Какой метод накапливал большую ошибку: дуги окружностей или ломанные? Почему?*
	\item Какие еще алгоритмы решения этой задачи вы можете предложить? В чем их преимущества и недостатки?
\end{enumerate}