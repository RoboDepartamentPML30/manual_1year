{\LARGE\bfseries Введение}\\\\

Последний этап данного курса~--- это выполнение самостоятельного проекта командой учащихся. Работая над проектом, помимо конкретных технических результатов дети учатся следующим универсальным умениям:

\begin{itemize}
	\item ставить стратегическую цель (отдаленную по времени, но значимую) и разбить ее на тактические шаги;
	\item оценивать имеющиеся ресурсы, в том числе собственные силы и время, распределять их;
	\item добывать информацию, критически оценивать ее, ранжировать по значимости, ограничивать по объему, использовать различные источники, в т.ч. людей, как источник информации;
	\item планировать свою работу;
	\item выполнив работу, оценить ее результат, сравнить его с тем, что было заявлено в 
	\item качестве цели работы;
	\item увидеть допущенные ошибки и не допускать их в будущем.\\\\
\end{itemize}

В рамках нашего курса учащимся предлагается групповая форма проектной деятельности, имеющая следующие преимущества:

\begin{itemize}
	\item в проектной группе формируются навыки сотрудничеств;
	\item проект может быть выполнен наиболее глубоко и разносторонне;
	\item на каждом этапе работы, как правило, есть свой ситуационный лидер. Каждый, в зависимости от своих сильных сторон, включается в работу на определенном этапе;
	\item в рамках проектной группы могут быть образованы подгруппы, предлагающие различные пути решения проблемы, идеи, гипотезы, точки зрения. Элемент соревновательности повышает мотивацию и позитивно сказывается на результате.\\\\
\end{itemize}

На заключительном занятии кружка проводится защита всех проектов в формате  научной конференции с докладами. Учащимся будет предложено проанализировать свою деятельность и деятельность своих товарищей (подробнее см. Занятие~\ref{lesson29}).

После защиты учащиеся могут выступить со своими проектами на соревнованиях и конференциях различного уровня  (список см. Приложение) и получить внешнюю оценку своей деятельности.

Строение занятий 23--29 немного отличается от других занятий в данном пособии. По умолчанию во всех этих занятиях ведется самостоятельная работа команд по проектам. Каждое занятие выделяет определенный этап в развитии проекта учащихся, позволяет преподавателю подробнее остановиться на анализе различных аспектов проектной деятельности. Важно отметить, что в зависимости от уровня подготовки группы каждый этап может потребовать не одного, а двух или даже трех занятий. 

В каждом занятии есть основные результаты данного этапа, конкретный нюанс проектный деятельности на который следует обратить внимание и примеры деятельности учащихся на данном этапе. Часть разделов написаны только  для преподавателя, для его подготовки к занятиям этого этапа,  в таком случае в плане урока им выделено 0 минут. 

Главное, от чего следует отталкиваться~--- это деятельность конкретных команд на занятии, всесторонняя помощь и направление учащихся.\\\\
~\\\\
~\\\\
\noindent Авторы:\\\\
\indent Лузина Екатерина Павловна\\
\indent Лузин Дмитрий Валерьевич\\
\indent Крылов Георгий Андреевич\\