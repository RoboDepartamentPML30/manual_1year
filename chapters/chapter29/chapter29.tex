\chapter{\label{lesson29}Финальная защита проектов}
{\bfseries Анонс:}\\\\
Принципы составления итоговой документации по инженерному проекту. Принципы оформления отчетов. Финальная защита проектов.\\\\
{\bfseries Цели:}
\begin{itemize}
	\item{}{\bfseries Обучающие:} Проверить уровень подготовленности учащихся по  изученным темам, закрепить умения и навыки публичных выступлений.   
	\item{}{\bfseries Развивающая:} Сформировать умение анализировать на основе нескольких источников, умение прослеживать причинно-следственные связи и выявлять признаки понятий,    развить умение извлекать знания из различных источников,    умение планировать свою деятельность.\\
\end{itemize}	
{\bfseries Ход занятия:}\\\\
\begin{tabular}[h!]{lll}
	{\hyperlink{lesson29x1}{1. Организационный момент}}&{Презентация}&{(5 мин)}\\
	{\hyperlink{lesson29x2}{2. Структура документации}}&{Практика}&{(10 мин)}\\
	{\hyperlink{lesson29x3}{3. Оформление документации}}&{Практика}&{(10 мин)}\\
	{\hyperlink{lesson29x4}{4. Защита проектов}}&{Обсуждение}&{(90 мин)}\\
\end{tabular}\\\\

{\hypertarget{lesson29x1}{\blackBlueText{I. Организационный момент}}}\\\\

Последняя неделя посвящена подготовке к защите проектов и написанию технической документации. В рамках подготовки к защите рекомендуется еще раз кратко повторить принципы публичных выступлений и написания презентаций, изложенные в Занятии 4. При высоком уровне тревожности детей и малой практике выступлений, можно провести тренировочные выступления, обкатать и поправить речи докладчиков.

Если учащимся тяжело давалась формализация своих мыслей в формате технической книги, то написание технической документации можно отложить до следующих ступеней и ограничиться подготовкой и финальной защитой. Если же учащиеся уверено справлялись с нагрузками, то к защите им предлагается подготовить для жюри так же и техническую документацию.

Техническая книга~--- документ внутренний, дневник, рабочая тетрадь. В ней отображаются мысли, тупиковые ветви работы, идеи, пожелания. Техническая документация проекта~--- итог вей работы, предельно краткое и сжатое изложение результатов труда. Можно сказать сухой остаток технической книги, который можно предложить для беглого ознакомления с проектом любому постороннему человеку. Важно обратить внимание учащихся на то, что робот будет рано или поздно разобран, их объемную техническую книгу новое поколение читать вряд ли станет, таким образом, именно техническая документация остается «в наследство» последующим кружковцам. Именно по ней можно будет воспроизвести их робота и спустя 10 лет. Итак, как оформит такой документ?\\\\

{\hypertarget{lesson29x2}{\blackBlueText{II. Структура документации}}}\\\\

Структура технического документа по проекту довольно проста:

\begin{enumerate}
	\item Цель
	\item Задачи
	\item Выбранное решение для каждой задачи и  его аргументация
	\item Результаты 
	\item Заключение 
\end{enumerate}

Остановимся подробнее на каждом из пунктов. Цель проекта конкретна, позволяет четко оценить достигнута она или нет и по сути является сжатым пересказом задания. Какого именно робота, обладающего какими качествами, решающего какие задачи и удовлетворяющего каким критериям, мы хотим построить.

Пример: Создать автономный автобус, двигающийся по чёрной линии, нанесённой на белое поле. При попадании автобуса в зоны жёлтого цвета, он останавливается, открывает двери, ждёт, пока зайдут пассажиры, закрывает двери и продолжает движение. Автобус должен быть устойчив к неровностям дороги, лёгким землетрясениям и погодным условиям.

Задачи вытекают из цели и по сути представляют собой краткий перечень проблем, которые необходимо решить для достижения цели. Каждое из предыдущих занятий решало свою задачу или несколько, так что при должной фиксации творческого процесса в технической книге, задачи можно просто переписать оттуда.

Пример:

\begin{enumerate}
	\item Создать конструкцию автоматически открывающихся дверей, работающих от одного мотора.
	\item Создать корпус автобуса внутренними размерами не менее 10х10х10 см из одного набора Lego Mindstorms NXT 2.0.
	\item Обеспечить высокий дорожный просвет и угол продольной проходимости автобуса.
	\item Протестировать конструкцию на прочность. 
	\item Написать программу движения по линии для созданного автобуса. 
	\item Написать программу остановки в желтых зонах и открытия дверей.
\end{enumerate}

Если в задачи технической книги входило подробное описание всех рассмотренных вариантов решения задач, то в итоговой документации следует зафиксировать лишь конечные решения с кратким перечнем его сильных сторон.

Пример: Для движения по линии использован пропорциональный регулятор. Принцип его работы следующий\dots Его преимущества перед релейным регулятором\dots
Результаты при качественном выполнении проекта есть пересказ целей в прошедшем времени. «Реализовать»~--- «реализовано», «создать»~--- «создано», «придумать»~--- «придумано». Все поставленные цели должны быть проанализированы с точки зрения выполнены они или нет, и если нет, то почему, что помешало.

Пример: Была создана модель робота-автобуса, который способен  автономно передвигаться по чёрной линии, нанесённой на неровное белое поле, останавливаться в специальных зонах, открывать и закрывать двери. 

Заключение подводит итог всей проделанной работе. Оно может содержать мысли авторов о том, чему они научились, какие навыки приобрели. Так же стоит коснуться вопросов дальнейшего развития проекта и применения  данного робота в реальной жизни. Возможен технический анализ каких-то отраслей робототехнике в целом и изобретенной конструкции в частности.

Пример: Данная модель предназначена для изучения возможностей автономного передвижения транспорта. Возможно, в будущем такие автобусы будут ходить по городам, на производствах, в выставочных комплексах, торговых центрах.
Осталось перечислить достоинства использования таких автобусов:
\begin{enumerate}
	\item Экономия топлива и улучшение экологии (автобусы будут электрические)
	\item Уменьшение количества аварий (отсутствие человеческого фактора, усталости водителя)
	\item Высокая проходимость
	\item Уменьшение пробок (все автобусы будут ходить по специально выделенным полосам и с определённым интервалом) 
	\item Безопасное передвижение ночью 
	\item Возможность использовать модели небольших размеров в помещениях.
\end{enumerate}

{\hypertarget{lesson29x3}{\blackBlueText{III. Оформление документации}}}\\\\

Помимо логической структуры технической документации кратко затронем вопрос оформления. Удобно раздать эти рекомендации в качестве памятки детям (см. Приложение).

\begin{enumerate}
	\item У документа должен быть титульный лист. На середине листа указывается название проекта, состав команды и руководитель. Наверху листа указывается название учебного объединения, а внизу~--- город и год.
	
	\item Новая мысль~--- новый абзац. Причем новый абзац начинается с красной строки с отступом. Отступы делаются не множеством пробелов, а одним нажатием клавиши Tab.
	
	\item Размер, тип шрифта, межстрочный интервал могут быть обговорены отдельно. Если четких указаний не поступило и есть сомнения~--- всегда выбирайте Times New Roman 12, не ошибетесь. Если сомнений нет – выбирайте сами. Но в любом случае не стоит задирать размер шрифта больше 14 и делать его меньше 8.
	
	\item Текст выравнивается по ширине страницы. Иллюстративный материал выравнивается по середине.
	
	\item Все рисунки, таблицы, графики, схемы должны иметь подписи. Подпись имеет вид: Тип (Рис., Таб.) Номер ( 1,2,3..) Описание ( Вид кабины автобуса сбоку, Таблица диаметров колес и скоростей движения). 
	
	\item Большие объемы дополнительной информации (8 ракурсов модели робота, полный текст программы) оформляются в виде приложений.
	
	\item Присутствует список использованной литературы (допустимы ссылки на интернет ресурсы).\\\\
\end{enumerate}

{\hypertarget{lesson29x4}{\blackBlueText{IV. Защита проектов}}}\\\\

Итак, пришло время финальной защиты проектов. По возможности обставьте это мероприятие серьезно и празднично, пригласите сторонних преподавателей, специалистов, инженеров. Очередность выступления можно определить при помощи жеребьевки. Заранее обговорите время, предоставляемое каждому докладчику для выступления. Рекомендуемый план выступления:

\begin{enumerate}
	\item Доклад  (15 минут)
	\item Демонстрация работы робота (3 минуты)
	\item Вопросы учащихся (10 минут)
	\item Вопросы жюри (10 минут)
	\item Коллективное обсуждении (10 минут)
\end{enumerate}

{\slshape В основном обсуждение будет касаться конкретных технических решений, использованных в том или ином проекте, но обязательно надо уделить время обсуждению общих моментов, на которых акцентировалось внимание в ходе работы:
	
\begin{enumerate}
	\item  Сформулировал ли докладчик цели и задачи проекта?
	\item Достигла ли команда заявленных целей?
	\item Какие источники информации использовала команда? Как оценивалась их достоверность?
	\item Как были распределены роли в команде?
\end{enumerate}}

По окончании выступления наградите всех участников грамотами, особо отличившимся командам можно вручить кубки.  Так же можно провести «неофициальное» обсуждение прошедшего мероприятия.