\chapter{Работа со стандартными объектами}
{\bfseries Анонс:}\\\\
Работа со стандартными объектами: банки, мячи. Системы обнаружения, захвата, анализа цвета, перемещения.\\\\
{\bfseries Цели:}
\begin{itemize}
	\item{}{\bfseries Обучающие:} Познакомить учащихся с основными типами работы с внешними объектами. 
	\item{}{\bfseries Развивающая:} Развитие у учащихся таких качеств как самостоятельность, трудолюбие, аккуратность, собранность.\\
\end{itemize}	
{\bfseries Ход занятия:}\\\\
\begin{tabular}[h!]{lll}
	{\hyperlink{lesson21x1}{1. Организационный момент}}&{Презентация}&{(20 мин)}\\
	{\hyperlink{lesson21x2}{2. «Царь горы»}}&{Игра}&{(90 мин)}\\
	{\hyperlink{lesson21x3}{3. Анализ технических решений}}&{Обсуждение}&{(15 мин)}\\
\end{tabular}\\\\

{\hypertarget{lesson21x1}{\blackBlueText{I. Организационный момент}}}\\\\

Одной из важнейших конструкторских задач робототехники является задача манипуляции с внешними объектами. В реальности это могут быть детали автомашин, собирающихся на конвейере, если говорить о промышленных роботах; боевое оружие, если говорить о военных роботах; или ваша рука, для робота-андроида. Задача создания высокоточных манипуляторов со множеством степеней свободы очень сложна, поэтому в школьных соревнованиях ее обычно максимально упрощают. Однако во множестве спортивных дисциплин надо уметь обнаруживать препятствия, распознавать цвет и форму объектов, измерять расстояние до него. 

Так,  одна из задач всемирной олимпиады роботов (WRO) в 2013 году, моделировала процесс сборки яиц варана на острове Комодо. Робот должен был двигаться по пересеченной местности и собирать себе в накопитель шарики одного цвета, и не трогать шарики другого цвета, а так же производить подсчет числа шариков.
Данное занятие посвящено работе со стандартными объектами соревнований :с банками 0.33 л из-под газировки разных цветов и шарами разного цвета из дополнительного набора Лего (9797 LEGO MINDSTORMS Education NXT Base Set).\\\\

{\hypertarget{lesson21x2}{\blackBlueText{II. «Царь горы»}}}\\\\

Учащимся предлагается разбиться на команды по 2--3 человека. Каждой команде соответствует цветной магнит, изначально все они находятся у подножья «горы» на доске. Всем командам выдается по карточке с первым заданием. Далее начинается скоростной подъем~--- выполнив первое задание, команда получает следующее и поднимается на один шаг вверх по горе. Выигрывает команда выше всех забравшаяся за полтора часа.\\\\

Задания:

\newcounter{competitionKingHill}
\begin{itemize}
	\renewcommand{\labelitemi}{\stepcounter{competitionKingHill}\thecompetitionKingHill)}
	\item В круге стоит банка. Робот располагается в центре круга. Ему необходимо автономно обнаружить банку и вытолкать ее за пределы круга.
	\item В круге стоит черная и белая банка. Робот располагается в центре круга.  Ему необходимо вытолкать черную банку за пределы круга и не тронуть белую.
	\item В круге стоит банка. Робот располагается в центре круга. Ему необходимо автономно обнаружить банку и вынести банку за пределы круга, не коснувшись пола.
	\item В круге на подставке находится шарик. Робот располагается в центре круга. Ему необходимо забрать шарик с одной подставки и перенести ее на другую за пределами круга,  не коснувшись пола.
	\item В круге на подставках находятся два шарика. Робот располагается в центре круга. Ему необходимо забрать шарик с одной подставки и перенести ее на другую,  не коснувшись пола и не сдвинувшись с места.\\\\
\end{itemize}

{\hypertarget{lesson21x2}{\blackBlueText{III.Анализ технических решений}}}\\\\

После окончания соревнования необходимо проанализировать все технические проблемы, возникавшие решения и возможные решения. Задачи, возникавшие перед командами:	

\newcounter{tasksKingHill}
\begin{itemize}
	\renewcommand{\labelitemi}{\stepcounter{tasksKingHill}\thetasksKingHill)}
	\item Задача обнаружения банки.
	
	В случае, если спецкурс не располагает лазерным датчиком расстояния~--- единственный вариант поиска банки, это сонар. Робот вращается на месте в центре круга, пока не «увидит» что-то, на расстоянии меньше чем радиус круга.\\
	Можно так же обсудить преимущества вертикальной установки сонара. 
	\item Задача перемещения банки.
	
	Перемещение банки в контакте с полом так же обычно решается учащимися однозначно~--- робот едет прямо на банку и толкает ее своим бампером.
	Можно обсудить различные конфигурации бамперов, углы под которыми их можно повернуть. Ширина бампера должна быть не меньше ошибки в обнаружении сонаром местоположения банки.\\
	Можно обсудить различные конфигурации бамперов, углы под которыми их можно повернуть. Ширина бампера должна быть не меньше ошибки в обнаружении сонаром местоположения банки.
	\item  Задача определения цвета банки в монохромном режиме.
	
	Для определения черная банка или белая разумно использовать датчик освещенности или датчик цвета в режиме освещенности. Из общих соображений понятно, что для определения цвета необходимо подъехать как можно ближе к банке, но не уронить ее. Здесь следует вспомнить алгоритмы борьбы с инерцией робота (Занятие \ref{lesson10}). В качестве развития задачи можно предложить учащимся построить график показаний датчика от расстояния до черной банки и по графику найти  критическое расстояние, при котором датчик уже верно определяет цвет банки. Для этого понадобиться типовая программа тестирования датчика (Занятие \ref{lesson14}). Так же можно обсудить от чего еще зависит это расстояние, как уменьшить влияние внешней освещенности.
	\item Задача захвата банки и ее перемещения «на весу».
	
	Эта задача уже может вызвать различные варианты конструкций, но наиболее популярной является захват-клешня. Легкая банка не представляет трудностей в удержании, но интересным дополнительным исследованием может быть максимальная удерживаемая конструкцией масса. Кроем того надо заострить внимание на вопросе подъема банки от пола, как и за счет чего это происходит.
	\item Задача обнаружения шара.
	
	Шар, даже с учетом подставки сильно ниже банки, что заставляет тщательнее подходить к расположению сонара. Следует обсудить, как конструкционно добиться наинизшего положения.
	\item Задача захвата шара и его перемещения.
	
	Захват шара и его удержание весьма с ложны с конструкционной точки зрения, из клешни шар скорее всего выскользнет. Одним из наиболее удачных вариантов автору представляется прямоугольная каркасное хранилище для шаров, которое просто подхватывает шар своей нижней частью с подставки и закрывает его в хранилище.
	\item  Задача определения цвета шара в цветном режиме.
	
	Во-первых, это задача~--- первая в которой учащиеся будут работать с датчиком цвета в режиме определения цвета, а не в режиме освещенности. Для того, что бы правильно написать соответсвующую часть кода им потребуется поработать с Help (Занятие \ref{lesson11}).  Во-вторых,  возникнет вопрос расположения датчика цвета, т.к. над сонаром его уже не поместить, он окажется выше шара.
\end{itemize}