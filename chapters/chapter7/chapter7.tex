\chapter{Основные моменты устройства автомобиля}
{\bfseries Анонс:}\\\\
Самостоятельные доклады учащихся по различным узлам автомобиля.\\\\
{\bfseries Цели:}
\begin{itemize}
	\item{}{\bfseries Обучающие:} Раскрыть основные моменты устройства автомобиля: дифференциал, реечное рулевое управление, ручная и автоматические коробки передач.
	\item{}{\bfseries Развивающая:} Развивать умение извлекать знания из различных источников, речь учащихся, интерес к изучаемому предмету. Привить навыки самообразования.\\
\end{itemize}	
{\bfseries Ход занятия:}\\\\
\begin{tabular}{lll}
	\hyperlink{lesson7x1}{1. Реечное рулевое управление} & Презентация & (15 мин)\\
	\hyperlink{lesson7x2}{2. Дифференциал} & Презентация & (15 мин) \\
	\hyperlink{lesson7x3}{3. Ручная коробка передач} & Презентация & (15 мин) \\
	\hyperlink{lesson7x4}{4. Автоматическая коробка передач} & Презентация & (10 мин)\\
	\hyperlink{lesson7x5}{5. Принципы оформления презентаций} & Рефлексия & (20 мин)\\
	\hyperlink{lesson7x6}{6. Принципы публичных выступлений} & Рефлексия & (20 мин)\\
\end{tabular}\\\\

{\hypertarget{lesson7x1}{\hypertarget{lesson7x2}{\hypertarget{lesson7x3}{\hypertarget{lesson7x4}{\blackBlueText{I--IV. Реечное рулевое управление. Дифференциал. Ручная коробка передач.  Автоматическая коробка передач.}}}}}}\\\\

Темы докладов для учащихся подобраны таким образом, что бы повторить основные характеристики движения по окружности движение по окружности и зубчатых передач.  Непосредственно устройство дифференциала, рулевого управления и коробок передач в дальнейшем курсе использоваться не будут, литература по предмету широко доступна, поэтому данный урок не включает в себя материал по самим механическим системам. В принципы, темы могут быть скорректированы или заменены на другие по усмотрению преподавателя. Основная цель данного занятия (можно оформить его как мини конференцию с программками и докладчиками ) – повторить материал предыдущих занятий, показать тесную связь этой теории с реальными, окружающими детей механизмами и развить навыки публичных выступлений у учащихся.\\\\

{\hypertarget{lesson7x5}{\blackBlueText{V. Принципы оформления презентаций}}}\\\\

\begin{flushright}
	«Идеальная презентация~--- это десять слайдов тридцатым шрифтом».
\end{flushright}

Сегодня, редкое публичное выступление обходится без презентации. Презентации делают пятиклассники для уроков и академики для докладов. Самый простой и распространенный редактор Microsoft PowerPoint интуитивно осваивается за 20 минут и дает огромные возможности по формам представления информации. Тем не менее, очень многие докладчики, вне зависимости от возраста и статуса, совершают досадные ошибки в составлении презентаций, на которые хотелось бы обратить внимание.

Главное, что нужно осознать: презентация лишь дополняет вашу речь, основное внимание слушателей должно быть сосредоточено на докладе. В хорошей презентации нет больших объемов текста - их долго, неудобно читать и это отвлекает слушателя. В хорошей презентации нет дополнительной информации, не связанной с текстом доклада. В хорошей презентации нет выплывающих, выпрыгивающих, выцветающих блоков, если вы, конечно, презентуете не карнавальное агентство.

Что же есть в этой загадочной хорошей презентации? Титульный и финальный слайд. На титульном слайде положено представить название своей работы, автора, руководителя и учебное заведение/объединение. На финальном слайде поблагодарите слушателей за внимание. Единое деловое оформление всех слайдов. Схемы, блок-схемы, диаграммы, графики, рисунки. Смысл слайдов в том чтобы дать визуальный образ вашей речи, внутренних связей между предметами, причин и следствий.

Из вышесказанного очевидно, что докладчик ни в коем случае не должен читать свою презентацию. Расставляя смысловые акценты в своей речи, он иллюстрирует ключевые моменты на слайдах, стоя лицом к аудитории, а не к слайду.\\\\

{\hypertarget{lesson7x6}{\blackBlueText{VI. Принципы публичных выступлений}}}\\\\	

\begin{flushright}
	«Если вы не можете объяснить это просто - значит, вы сами не понимаете этого до конца»	
	
	А.Энштейн.
\end{flushright}

Ораторское мастерство – это целое искусство, которому посвящены тысячи томов, учебных курсов и тренингов. Не имея целью полно осветить все сложности этого искусства, хочется остановиться на нескольких наиболее простых и часто встречающихся проблемах учащихся. 

Главная проблема~--- это, конечно, полное не владение материалом доклада. К сожалению, отличная изначально идея рефератов и докладов учащихся сведена зачастую к «загуглил--скопировал--прочитал по бумажке-получил свою пять». Многие впервые начинают читать свой реферат уже на занятии.
Первое с чего нужно начать~--- это максимально мотивировать учащихся. Заинтересовать их темами доклада, поставить проблемные вопросы, добавить игровой момент («настоящая» конференция с докладчиками, трибуной, программой) и соревновательный (листы оценки и самооценки, поощрение за лучшее выступление). 

К сожалению, одной мотивации недостаточно, нужно уметь работать с текстом и готовиться выступать. Проверка «понимаю ли я то, о чем собираюсь говорить?» проста, нужно попробовать пересказать текст своими словами. Получилось? Теперь надо попробовать пересказать текст кому-то из знакомых или родственников, абсолютно не разбирающихся в вопросе. Достигнув понимания материала, следует так же проверить наличие у доклада внутренней структуры. Любое выступление имеет введение, заключение и основную часть, разбитую на логические блоки, взаимосвязанные между собой.

Итак, есть ответы на вопрос «что я хочу сказать?» и «как я хочу это сказать?». Но остается не менее сложная задача – действительно сказать это перед аудиторией. Даже самая логичная, структурированная и умная речь может быть озвучена так, что ничего не будет понятно.  Рекомендации будут очень просты, но отработке каждого пункта стоит уделить внимание.
\begin{itemize}
	\item Говорить надо громко. {\slshape Дайте каждому ребенку возможность громко что-то сказать.}
	\item Говорить надо с интонациями и паузами. {\slshape Прочитайте отрывок текста из любого доклада с интонациями и паузами, а не монотонной бубнежкой под нос. Потренируйте детей выделять основную мысль в предложении и интонировать ее.}
	\item Не надо спешить. Протараторьте и произнесите нарочито медленно. {\slshape Обсудите убыстряется или замедляется темп речи при публично выступлении. Проведите опыт.}
	\item Не надо бояться запнуться, остановиться или посмотреть в доклад. {\slshape Поясните разницу между читать доклад и подглядывать в план. Потренируйтесь преодолевать запинки и оговорки.} 
\end{itemize}