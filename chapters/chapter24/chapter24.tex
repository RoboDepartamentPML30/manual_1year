\chapter{Цели и задачи проекта}
{\bfseries Анонс:}\\\\
Примеры и шаблоны написания технической книги. Формулировка целей и задач проекта.\\\\
{\bfseries Цели:}
\begin{itemize}
	\item{}{\bfseries Обучающие:} Проконтролировать усвоение принципов оформления технической книги. Разъяснить разницу между целями и задачами проекта и научить их формулировать. 
	\item{}{\bfseries Развивающая:} Развить структурное мышление и умение планировать свою деятельность.\\
\end{itemize}	
{\bfseries Ход занятия:}\\\\
\begin{tabular}[h!]{lll}
	{\hyperlink{lesson24x1}{1. Организационный момент}}&{Презентация}&{(5 мин)}\\
	{\hyperlink{lesson24x2}{2. Техническая книга}}&{Презентация}&{(20 мин)}\\
	{\hyperlink{lesson24x3}{3. Цели и задачи}}&{Презентация}&{(20 мин)}\\
	{\hyperlink{lesson24x4}{4. Свободное творчество}}&{Практика}&{(80 мин)}\\
	{\hyperlink{lesson24x5}{5. Итоги этапа}}&{Обсуждение}&{(10 мин)}\\
\end{tabular}\\\\

{\hypertarget{lesson24x1}{\blackBlueText{I. Организационный момент}}}\\\\

После первой рабочей встречи проектной группы им следует напомнить, что процесс их творчества должен быть зафиксирован в письменной форме. Необходимо собрать у учащихся эти технические книги с материалами, по предыдущему занятию и просмотреть. В случае если они отвечают всем необходимым критериям, можно сразу перейти к разбору целей и задач проектов.\\\\

{\hypertarget{lesson24x2}{\blackBlueText{II. Техническая книга}}}\\\\

Как правило, учащиеся начинают с разработки общей концепции робота, поэтому первая техническая книга, которую вы получите – непонятная картинка, олицетворяющая будущего робота. На ней с очень большой вероятностью не будет ни подписей отдельных элементов, ни пояснений по работе, ни тем более хода обсуждения.

На этом этапе, учащимся еще сложно привыкнуть структурировать свои мысли, поэтому  предложите им на первое время следующий шаблон отчета по занятию, который будет подшиваться к их технической книге:

\begin{enumerate}
	\item\underline{Дата и время начала собрания:}\\
	15.04.13\\
	16:00
	
	\item\underline{Поставленные задачи:}\\
	Собрать информацию по принципам работы реечного рулевого управления и дифференциала.
	
	\item\underline{Реализация задач:}\\ 
	Иванов Иван отвечал за поиск материалов по дифференциалу. Петров Петр изучил реечное рулевое управление.  За дополнительной информацией обращались к учителю физики Андрееву А.А.
	
	\item\underline{Результаты и итоги собрания:}\\
	Дифференциал~--- это\dots Его схема находится на рис\dots\\
	Реечное рулевое управление~---это\dots
	
	\item\underline{Идеи для следующих собраний:}\\
	Собрать модель дифференциала.
\end{enumerate}

{\slshape Оформите совместно с учащимися  в соответствии с шаблоном одну--две книги, по мотивам их деятельности за первую неделю.}\\\\

По мере освоения детей формата записей в техническую книгу, можно будет ослабить формальные критерии и допустить вольный способ изложения. Важно, что бы в книге по-прежнему присутствовали ответы на следующие вопросы (их можно раздать в качестве памятки, Приложение):

\begin{enumerate}
	\item Когда работали? (дата)
	\item Над чем работали? (задачи)
	\item Кто работал? (реализация задач)
	\item Какие идеи высказывались? (реализация задач)
	\item Откуда брали информацию? (реализация задач)
	\item Как и почему приняли конкретное решение?
	\item В чем состоит решение? (результаты и итоги)
	\item Как собирали? (реализация задач)
	\item Кто собирал? (реализация задач)
	\item Что собрали? Фото, рисунки, видео, модели. (результаты и итоги)
	\item Каковы дальнейшие планы? (идеи для следующих собраний)
\end{enumerate}

Первая нешаблонная техническая книга, которая была получена проекту Парковка  выглядела так (орфография и пунктуация авторские):\\\\


{\slshape\begin{flushright}
		День 1(09.04.13)
	\end{flushright}
	
	\noindent Сегодня я получил от руководителя кружка Екатерины Павловны задание построить робота, способного автономно парковаться вдоль стенки.\\\\
	Требования к роботу:
	
	\begin{itemize}
		\item автономность.
		\item параллельная парковка задним ходом в нишу шириной 30 см и глубиной 17 см.
		\item использование в конструкции робота реечного рулевого управления и дифференциала на оси ведущих колес.
		\item максимальные габариты робота не должны превышать 15х25 см.
		\item стойкость к тряске и др. внешним механическим воздействиям.
	\end{itemize}
	
	\noindent Первым делом я пришел домой и узнал в интернете, что такое дифференциал и реечное рулевое управление (далее РРУ).\\\\
	Вот определения этих терминов:
	
	\begin{itemize}
		\item Дифференциал~--- устройство, позволяющее перераспределять мощность, подаваемую на ось мотора, между двумя ведущими колесами. Используется для предотвращения излишнего стирания резины при выполнении машиной поворота.
		\item РРУ`--- устройство для поворота колес автомобиля, представляющее из себя две рейки, на которых закрепляются колеса, одна из которых неподвижна, в то время как другая двигается параллельно ей и поворачивает колеса.
	\end{itemize}
	
	После этого я построил в LDD модель дифференциала (рис. 1) и РРУ (рис. 2)}\\\\

Посмотрим, насколько этот образец соответствует предъявленным требованиям. Дата присутствует. Над чем велась работа понятно из контекста~--- производился первичный поиск информации, уточнение формулировки задания. А вот с авторством проблемы~--- кто тот  «я», который искал информацию в интернете, не уточняется. Так же неудачна и сама формулировка  источника информации: «в интернете». Легко доказать, что дальнейшие сведения про дифференциал представляют собой точную цитату с определенного ресурса, в таком случае обязательно указывать точный источник (ссылку). 
Так же плохо освещен вопрос о сборке моделей дифференциала  и РРУ. Не понятно как определения этих устройств помогают понять их устройство, очевидным образом использовались еще какие-то не указанные материалы. По непонятной причине сами рисунки моделей так же  не предоставлены. Нет следов планирования деятельности на следующий раз.\\\\

{\hypertarget{lesson24x3}{\blackBlueText{III. Цели и задачи}}}\\\\

На начальном этапе очень важно выделить сформулировать главную цель проекта и выделить задачи, которые необходимо решить для ее достижения. Эти понятия часто путают, поэтому кратко остановимся на их основных свойствах и различиях.

Цель~--- это конечный желаемый результат. Задача~--- проблемная ситуация с явно и заранее заданной целью. Любая конкретизация, детализация цели позволяет ей распасться на набор задач. Задача содержит в себе сроки и ресурсы. Задача – это единичное действие. Цель~--- это ответ на вопрос «Чего я хочу достигнуть?». Задачи – это ответ на вопрос «Что мне сделать, что бы этого достигнуть?». Задача~--- это средства реализации цели.

Проиллюстрируем сначала бытовым примером. Цель: стать богатым. Задачи: выбрать высокооплачиваемую работу, выучиться на специалиста в этой области, получать повышения каждые два года. Или найти богатого мужа/жену, обольстить, жениться/выйти замуж, составить удачный брачный договор. Или изобрести что-то новое, получить патент, выгодно продать патент одной из ведущих компаний. Или найти в архивах сведения о старых кладах, купить металлодетектор, найти клад. Видим, что к одной и той же цели в принципе могут вести разные группы задач, но они всегда понятны и конкретны.

{\slshape Придумайте совместно с детьми еще несколько целей и распишите различные последовательности задач, ведущие к их решению.}

То же самое на примере робототехнических проектов. Цель: построить автономного паркующегося четырех колесного робота с РРУ и дифференциалом. Задачи: разобраться с устройством дифференциала и РРУ, реализовать эти устройства, придумать алгоритм парковки, реализовать алгоритм парковки.\\\\

{\hypertarget{lesson24x4}{\blackBlueText{IV. Свободное творчество}}}\\\\

На втором занятии учащиеся уже могли создать какие-то рабочие узлы и модели будущего робота, а так же столкнуться  с заложенными в задании трудностями. Важно не только формально сформулировать задачи для каждой команды, но и обсудить как они проистекают из тех проблем, с которыми они уже столкнулись, какой у команды план решению поставленных задач.\\\\

{\hypertarget{lesson24x5}{\blackBlueText{V. Итоги этапа}}}\\\\

Общие результаты второго этапа, вне зависимости от конкретной задачи, должны выглядеть так:

\begin{itemize}
	\item Сформулированы цели и задачи проекта.
	\item Сформировано представление об общей концепции робота.
	\item Проведен первичный поиск и знакомство с литературой.
	\item Выработана система разделения труда.
	\item Учащимся понятны принципы написания технической книги.
\end{itemize}