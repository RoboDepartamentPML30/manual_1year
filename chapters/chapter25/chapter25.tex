\chapter{\label{lesson25}Работа с информацией}
{\bfseries Анонс:}\\\\
Обсуждение технических решений. Тренинг по работе с информацией.\\\\
{\bfseries Цели:}
\begin{itemize}
	\item{}{\bfseries Обучающие:} Объяснить принципы коллективного обсуждения различных технических решений. Закрепить принципы успешного межличностного взаимодействия. 
	\item{}{\bfseries Воспитательная:} Содействовать воспитанию серьезного отношению к учебному труду, развитие культуры речи. \\
\end{itemize}	
{\bfseries Ход занятия:}\\\\
\begin{tabular}[h!]{lll}
	{\hyperlink{lesson25x1}{1. Организационный момент}}&{Презентация}&{(5 мин)}\\
	{\hyperlink{lesson25x2}{2. Конструкционная задача}}&{Презентация}&{(20 мин)}\\
	{\hyperlink{lesson25x3}{3. Работа с информацией}}&{Игра}&{(20 мин)}\\
	{\hyperlink{lesson25x4}{4. Свободное творчество}}&{Практика}&{(70 мин)}\\
	{\hyperlink{lesson25x5}{5. Итоги этапа}}&{Обсуждение}&{(10 мин)}\\
\end{tabular}\\\\

{\hypertarget{lesson25x1}{\blackBlueText{I. Организационный момент}}}\\\\

Если проектных групп не очень много, рекомендуется провести обсуждение первой задачи конструкционной части с каждой группой по отдельности, работая с конкретными задачами каждого проекта.

Практикум по работе с информацией можно провести в каждой проектной группе или только в группах, нуждающихся в этом, на взгляд преподавателя. Для практикума потребуется заранее подготовить листы с информацией, напечатанные на картонных карточках.\\\\

{\hypertarget{lesson25x2}{\blackBlueText{II. Конструкционная задача}}}\\\\	

Конструкционную часть задачи Автобус была разбита на две составляющие: корпус и механизм открывания дверей. Какие ограничения на корпус накладывает техническое задание?

\begin{itemize}
	\item Внутрь автобуса должен помещаться « пассажир» размером  10х10х10 см.
	\item Разрешено использовать датчики и моторы только одного набора Lego Mindstorms NXT 2.0 и один дополнительный датчик освещенности.
	\item Запрещено использование любых клейких, липких материалов в конструкции робота.
	\item Робот должен быть устойчив к любым погодным катаклизмам (сильному ветру, сотрясениям почвы и т.п.)
	\item Робот должен быть приспособлен к езде по российским дорогам (на пути возможны ямы, колдобины и т.п., размером не более 20 мм)
\end{itemize}

Беглая работа с набором выявляет первую, заложенную в задачу, проблему – количество разрешенных к использованию деталей недостаточно для создания «сплошного» куба с внутренним размером хотя бы 10х10х10 см.
Команда должна зафиксировать эту проблему в технической книге и провести коллективное обсуждение путей решения. Могут быть предложены следующие варианты:

\begin{enumerate}
	\item Создание кубического каркаса из балок и обтягивание его бумагой/полиэтиленом.
	
	К минусам этого варианта можно отнести возможные проблемы с креплением покрова ( на липкие материалы наложен запрет), такой робот будет очень непрочным и сильно парусить при ветре ( т.е. неустойчив к одному из перечисленных погодных катаклизмов).
	
	Достоинство такого варианта в том, что он расходует очень мало деталей Лего. Кроме того робот получается легким.
	
	\item Создание корпуса  без использовании Лего, из альтернативных материалов, таких как фанера, листы жести, пластик.
	
	Подобный вариант влечет за собой проблемы в  поиске и обработке материалов, а так же в соединении кабины и лего механизма по открыванию самих дверей. Помимо этого такой робот так же неутойчив при сильном ветре.
	
	К плюсам этого варианта стоит отнести высокую прочность и долговечность конструкции при качественном выполнении. 
	
	\item Создание ячеистого каркаса из балок.
	
	Недостатком этого варианта является большой расход деталей. Кроме этого пассажир может намокать, надо уточнять относится ли ливень к рассматриваемым погодным катаклизмам.
	
	Плюсом является малая парусность, сравнительно малый вес и большая прочность конструкции.\\\\	
\end{enumerate}

{\hypertarget{lesson25x3}{\blackBlueText{III. Работа с информацией}}}\\\\	

По каждому вопросу  команда ведет коллективное обсуждение за которым преподаватель должен время от времени наблюдать, помогать высказаться робким, подталкивать учащихся к выделению плюсов и минусов того или иного решения, инициировать поиск информации по вопросу, создание прототипных конструкций. 

При необходимости рекомендуется провести нижеследующее упражнение для отработки навыков коллективной работы с информацией.

Тренинг направлен на отработку умения делиться информацией, выслушивать ее и объединять для общей цели знания всех участников команды. Общая концепция тренинга следующая.  Детям раздаются листочки с частями информации. Каждый может видеть только свои листочки, обмениваться ими нельзя, но можно устно делиться информацией. У каждого есть информация полезная для решения общей задачи и бесполезная. 

Общая идея самого задания может быть, например, такова:  дан объем работы и производительность труда и надо найти время выполнения работы. Подобные задачи решаются, начиная с 5 класса, так что в простейшем варианте трудности для учащихся не представляют. Однако, такая задача хороша тем, что ее можно легко запутывать сколько угодно, добавляя новые и новые кусочки головоломки. 

Простейший вариант: робот состоит из 400 деталей, за день 1 учащийся изготавливает 5 деталей, сколько будут работать 3 учащихся над постройкой целого робота? Теперь каждый блок (объем работы, производительность, единицы измерения времени) можно начинать усложнять.
\begin{center}
	\begin{tabular}[h!]{|p{0.25\linewidth}|p{0.25\linewidth}|p{0.25\linewidth}|p{0.25\linewidth}|}
		\hline
		~ & Простой уровень & Средний уровень & Сложный уровень\\
		\hline
		Объем работы & Робот состоит из 400 шпунтиков и 20 балок. & {Робот состоит из 400 шпунтиков и 20 балок.
			
			Для прикрепления одной балки надо 5 жгутиков.} & Робот состоит из 400 шпунтиков и 20 балок.
		
		Для прикрепления одной балки надо 5 жгутиков и 1 прусик.
		
		Собранного робота надо покрасить.\\
		\hline
		Производительность & На изготовление шпунтика уходит 4 часа.
		
		На изготовление балки уходит в 3 раза больше времени, чем на изготовление шпунтика. & На изготовление шпунтика уходит 4 кварты.
		
		На изготовление балки уходит в 3 раза больше времени, чем на изготовление шпунтика.
		
		1 жгутик крепится 1 болл. & На изготовление шпунтика уходит 4 кварты.
		
		На изготовление балки уходит в 3 раза больше времени, чем на изготовление шпунтика.
		
		1 жгутик крепится 1 болл.
		
		Прусики выдаются только парами.
		
		Кладовая прусиков открыта только по  куатро.
		
		Покраска занимает 20 кварт и 84 болла.\\
		\hline
		Время & Рабочий день длится 8 часов.
		
		Каждый третий день~--- выходной. & В 1 кварте 7 боллов.
		
		Рабочий день длиться 35 боллов.
		
		В неделе 5 рабочих дней. & В 1 кварте 7 боллов.
		
		Рабочий день длиться 35 боллов.
		
		В неделе 5 рабочих дней.
		
		Один день сборщики отдыхают.
		
		Первый день  называется юн.
		
		Второй день называется дё.
		
		Третий день называется тре.
		
		Четвертый день называется куатро.
		
		Пятый день называется чинкуа.
		
		Шестой день называется отто.\\
		\hline
		Вопрос & Через сколько дней  закончится сборка робота? & На какой неделе закончится работа? & В какой день была закончена сборка?\\
		\hline
	\end{tabular}
\end{center}

На выполнение упражнения отводится фиксированное, достаточное, но не чрезмерное количество времени. По истечении времени группа должна дать ответ на поставленный вопрос. После этого производится разбор самой задачи и обсуждение взаимодействия участников. Рекомендуется обсудить следующие вопросы:

\begin{itemize}
	\item Как был организован обмен информацией?
	\item Каким образом выбиралась очередность выступления?
	\item Наблюдался ли в группе раскол, существовало ли несколько групп, которые независимо пытались искать ответ на поставленный вопрос?
	\item В каких случаях  параллельное решение задачи улучшает качество работы группы? Почему?
	\item Какие изменения в командное обсуждение вы теперь внесете?\\\\
\end{itemize}

{\hypertarget{lesson25x4}{\blackBlueText{IV. Свободное творчество}}}\\\\

В рамках этого этапа учащиеся заняты вполне конкретными конструкционными задачами, примеры обсуждения приведены, проведен практикум по улучшению командного взаимодействия. Задача учителя  - быть в каждый момент вовлеченным в работу детей, рекомендовать им некоторые пути решения их технических проблем, источники информации, но неизменно подчеркивать их личную ответственность за конечный результат, давать возможность совершить ошибки, но указывать на них.\\\\

{\hypertarget{lesson25x4}{\blackBlueText{V. Итоги этапа}}}\\\\

Общие результаты третьего этапа, вне зависимости от конкретной задачи, должны выглядеть так:

\begin{itemize}
	\item Сформулированы некоторые из проблем конструкционной части. 
	\item Рассмотрены различные пути решения проблем. Проведено обсуждение плюсов и минусов различных вариантов.
	\item Решена хотя бы одна проблема конструкционной части.
	\item Выработана система командного взаимодействия.
	\item Понятны и успешно реализуются принципы написания технической книги.		
\end{itemize}