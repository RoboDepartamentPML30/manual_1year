\chapter{Техническая документация}
{\bfseries\greenText{Анонс:}}\\\\
~\\\\
{\bfseries Цели:}
\begin{itemize}
	\item{}{\bfseries Обучающие:} Проконтролировать степень усвоения следующих основных знаний, умений и навыков, изученных и сформированных на предыдущих занятиях. Ознакомить с принципами оформления технической книги. 
	\item{}{\bfseries Развивающая:} Способствовать развитию  у школьников самостоятельности мышления в целях развития интеллектуальных способностей; умению переносить знания и навыки в новые ситуации.\\
\end{itemize}	
{\bfseries Ход занятия:}\\\\
\begin{tabular}[h!]{lll}
	{\hyperlink{lesson23x1}{1. Организационный момент}}&{Презентация}&{(25 мин)}\\
	{\hyperlink{lesson23x2}{2. Техническое задание}}&{Презентация}&{(0 мин)}\\
	{\hyperlink{lesson23x3}{3. Техническая книга}}&{Презентация}&{(20 мин)}\\
	{\hyperlink{lesson23x4}{4. Свободное творчество}}&{Практика}&{(80 мин)}\\
\end{tabular}\\\\

{\hypertarget{lesson23x1}{\blackBlueText{I. Организационный момент}}}\\\\

Итак, на сегодняшнем занятии каждая команда (2--3 человека) должна получить свое проектное задание, с формализованными критериями и целями. Наличие такого документа позволяет создать у детей ощущение «настоящего, как у взрослых» проекта, настроить их на рабочий лад  и работу с документацией, а так же впоследствии оценить соответствие целей и методов на защите.

Первое занятие по работе над проектом начинается с рассмотрения основных этапов работы над проектом и отчетности:

\newcounter{techBooK}
\begin{itemize}
	\renewcommand{\labelitemi}{\stepcounter{techBooK}\thetechBooK)}
	\item Знакомство с заданием. Создание и оформление концепции.
	\item Первые главы технической книги. Формулировка целей и задач.
	\item Реализация конструкционной части.
	\item Испытания. Создание модели в LDD.
	\item Реализация программной части.
	\item Создание технической документации.
	\item Защита проекта.
\end{itemize}

{\slshape Напомним, что каждому этапу посвящен отдельный урок, где на примере реализованных проектов,   будут фиксироваться основные достижения учащихся к этому моменту и разбираться возможные трудности.}

Итак, основные этапы затронуты, пришло время распределения между командами технических заданий. В случае выбора преподавателем однотипных технических заданий интересно провести лотерею/жеребьевку. В противном случае детям выдаются их индивидуальные задания.

После раздачи заданий детям дается некоторое время, что бы ознакомиться с ним и задать интересующие их вопросы. После этого внимание еще раз акцентируется на вопросах отчетности по проекту. Помимо итоговой защиты каждая команда пишет техническую книгу и сдает ее преподавателю.\\\\

{\hypertarget{lesson23x2}{\blackBlueText{II. Техническое задание}}}\\\\

Можно выделить три различных способа создания технического задания. Первый, наиболее естественный, но редко реализуемый, это формализовать идеи и мысли учащегося.  Как правило, дети с удовольствием придумывают фантастические конструкции (см. Практикум в Занятии 1),  но не хотят и не могут оформить свои предложения наукообразным образом. 

Если ребенок бросается от одной идеи к другой, большинство его предложений~--- это варианты воспроизведения роботов из голливудских фильмов или фантастических животных, то скорее всего он легко переключится на новый, предложенной преподавателем проект, если его правильно подать и поддерживать интерес к нему. В случае, когда ребенок много раз возвращался к одной и той же идее, пусть и немного наивной, предлагал новые, им придуманные концепции, разумным представляется помочь ему развить его мысль и построить его индивидуальный проект вокруг нее. К примеру, учащийся много раз порывавшийся построить человека-паука, просто паука, гусеницу, древолаза, в итоге обсуждения получил задание на робота-труболаза (см. Приложение).

Вторым способом создания технического задания для учащегося является упрощение реально существующего механизма. Такой способ имеет множество преимуществ. Во-первых, в ходе работы над проектом дети знакомятся с современными техническими решениями и расширяют собственную базу механизмов. Во-вторых, легко объяснить учащимся в какие реальные производственные задачи могут развиваться школьные проекты. В-третьих, легко сформировать разнообразные проекты схожей сложности всем учащимся. К заданиям этой группы относятся Автобус и Подъемник (см. Приложение).

К третьему типу технических заданий можно отнести задания, родившиеся из соревновательных задач.  Разумно выбирать такой вариант для детей, которым выступить на соревнованиях будет гораздо интереснее чем на конференции. В качестве технического задания немного перерабатывается, усложняется/упрощается регламент соревновательной дисциплины. Примером такого задания может служить Дорога-2.\\\\

{\hypertarget{lesson23x3}{\blackBlueText{III. Техническая книга}}}\\\\

Техническая книга отражает путь, проходимый командой при создании своего инженерного проекта. В ней фиксируются идеи, их достоинства и недостатки, формулируются возникшие проблемы, результаты испытаний и тестов робота, его принципиальные чертежи и детальные разработки.

Жюри на итоговой защите изучает книгу с тем, чтобы лучше понять, что собой представляет ваша команда, какой путь вы прошли от разработки до создания моделей роботов.

Команды могут вести записи в обычных тетрадях, в электронной форме или в сети. При оценке не делается различий между форматами ведения записей.

{\bfseries Электронная форма/Сетевое размещение:} Команды могут вести технические книги в электронной форме или в сети. Для предоставления в жюри техническая книга должна быть распечатана и сшита в папку размером не более чем 1,5". Все страницы должны располагаться по порядку и быть пронумерованы. От каждой команды требуется только одна копия книги.

{\bfseries Письменная форма:} Тетради со спиральным переплетом, лабораторные журналы или журналы для делопроизводства можно получить в школе, либо приобрести в местном канцелярском магазине. Воспользуйтесь следующими правилами для выбора типа книги:

\begin{enumerate}
	\item Не используйте книги с несшитыми листами. 
	\item Рекомендуется, чтобы в книге были нумерованные страницы для того, чтобы листы не могли быть заменены или удалены.
	\item От каждой команды требуется только одна копия книги.
	\item Разные команды не могут вести записи в одной технической книге.
\end{enumerate}

Техническая книга~--- это полная документация процесса разработки вашего робота. Она должна содержать эскизы, выдержки обсуждений с собраний, показывать, как происходила разработка и реализация, с какими препятствиями сталкивалась команда, описывать мысли каждого участника команды в течение всего периода работы. Некоторые рекомендации:

\begin{enumerate}
	\item Документируйте абсолютно все! 
	\item Книга должна быть организована так, чтобы посторонний человек мог составить мнение о команде и понять смысл вашей работы.
	\item Начните книгу с информации об участниках команды и кураторе, представив краткое резюме: имя, возраст (или класс в школе), интересы и причины, побудившие вступить в вашу команду.
	\item В начале каждого собрания начните заполнять новую страницу. Запишите дату, время начала и окончания собрания. Во время каждого собрания и в начале каждой страницы сделайте две колонки:
	\begin{enumerate}
		\item Колонка задач~--- Над чем работает команда и что узнает.
		\item Колонка размышлений~--- Туда участники команды записывают мысли о происходящем и возникающие вопросы.
	\end{enumerate}
	\item Записи должны вестись всеми участниками команды с указанием имени и даты.
	\item Записывайте все варианты конструкции робота и внесенные изменения непосредственно в книгу. Рекомендуется описывать все детально, включать в книгу эскизы. Выполняйте все расчеты в книге, а не на отдельных листах бумаги.
	\item Одним из примеров подробной документации являются фотографии и эскизы вашего робота.
\end{enumerate}

Ведение технической книги~--- постоянный процесс,  отображающий все изменения и развитие. В случае если вы ведете записи от руки, жюри не нужна «чистовая» версия книги, им нужна реальная вещь~--- с ошибками, помарками, поношенными и мятыми краями. Книга должна иметь рабочий вид! 

Передавая вашу книгу в жюри на турнире, при помощи клейкой бумаги для заметок отметьте 6--12 страниц, которые, по вашему мнению, характеризуют вас с лучшей стороны как команду. Судьи используют эти страницы на предварительном просмотре вашей книги.\\\\
\clearpage
{\hypertarget{lesson23x4}{\blackBlueText{IV. Свободное творчество}}}\\\\

Первое занятие~--- простор для творчества учащихся. Пусть они придумывают фантастические концепции решения своих задач, рисуют чертежи на салфетках и собирают красивые и бесполезные корпуса. Единственное, за чем надо следить, это что бы  был момент обсуждения, провоцировать ребят на диалог друг с другом. А так же, что бы все обсуждения и идеи в каком-то виде фиксировались в технической книге.