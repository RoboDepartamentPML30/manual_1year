\chapter{Сенсоры и регуляторы}
{\bfseries Анонс:}\\\\
Зачетное занятие по теме «Сенсоры и регуляторы».\\\\
{\bfseries Цели:}
\begin{itemize}
	\item{}{\bfseries Обучающие:} Проконтролировать степень усвоения следующих основных знаний, умений и навыков, изученных и сформированных на предыдущих занятиях.
	\item{}{\bfseries Развивающая:} Способствовать развитию  у школьников самостоятельности мышления в целях развития интеллектуальных способностей; умению переносить знания и навыки в новые ситуации.\\
\end{itemize}	
{\bfseries Ход занятия:}\\\\
\begin{tabular}[h!]{lll}
	{\hyperlink{lesson18x1}{1. Организационный момент}}&{Презентация}&{(5 мин)}\\
	{\hyperlink{lesson18x2}{2. «Поводырь»}}&{Игра}&{(90 мин)}\\
	{\hyperlink{lesson18x3}{3. Анализ технических решений}}&{Рефлексия}&{(20 мин)}\\
\end{tabular}\\\\

{\hypertarget{lesson18x1}{\blackBlueText{I. Организационный момент}}}\\\\

У каждой команды (2--3 человека) есть свое рабочее место, на каждые 3 команды приходится 1 испытательный полигон, с движущимся поводырем (схема в Приложении). Учащиеся могут пользоваться любой справочной литературой.\\\\
\clearpage
{\hypertarget{lesson18x2}{\blackBlueText{II. «Поводырь»}}}\\\\

\begin{center}
	Регламент соревнований\\
	{\bfseries Поводырь}
\end{center}

\noindent\orangeText{1. Условия состязания}
\begin{enumerate}
	\item За наиболее короткое время робот, следуя за роботом-проводником на расстоянии не более 30 см и не менее 15 см, должен добраться от места старта до места финиша.
	\item На прохождение дистанции дается максимум {\bfseries 1 минута}.
	\item {\bfseries Если робот будет отставать от проводника или опережать его более 5 секунд подряд, он будет дисквалифицирован.}	
	\item Во время проведения состязания участники команд не должны касаться роботов.
\end{enumerate}

\noindent\orangeText{2. Трасса}
\begin{enumerate}
	\item Проводник представляет собой белый цилиндр.
	\item Диаметр цилиндра~--- 160 мм.
	\item Высота~--- 200 мм.
	\item Максимальная скорость движения: 20 см/с.
	\item Траектория проводника~--- прямая черная линия.
	\item Линии старта/финиша~--- желтые.
\end{enumerate}

\noindent\orangeText{3. Робот}
\begin{enumerate}
	\item Максимальная ширина робота 40 см, длина~--- 40 см.
	\item Вес робота не должен превышать 10 кг.
	\item Робот должен быть автономным.
\end{enumerate}	

\noindent\orangeText{4. Правила отбора победителя}
\begin{enumerate}
	\item На прохождение дистанции каждой команде дается не менее двух попыток. В зачет принимается лучшее время из попыток.
	\item Победителем будет объявлена команда, потратившая на преодоление дистанции наименьшее время.
	\item Процедура старта: робот устанавливается участником на линии перед стартовой линией. До команды «СТАРТ» робот должен находиться на поверхности полигона и оставаться неподвижным. После команды «СТАРТ» участник должен запустить робота и быстро покинуть стартовую зону. Началом отсчета времени заезда является момент пересечения передней частью робота стартовой линии. Окончанием отсчета времени заезда является момент пересечения передней частью робота финишной линии.\\\\
\end{enumerate}		


{\hypertarget{lesson18x3}{\blackBlueText{III. Анализ технических решений}}}\\\\

\begin{enumerate}	
	\item Были ли у вас проблемы связанные с конечным временем реакции робота на изменения скорости помехи? Как вы с ними боролись?
	\item  Какой регулятор вы использовали? Почему?
	\item  Какие еще алгоритмы решения этой задачи вы можете предложить? В чем их преимущества и недостатки?
\end{enumerate}