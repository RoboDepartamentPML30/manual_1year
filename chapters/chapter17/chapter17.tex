\chapter{\label{lesson17}Соревнования «Линия-2»}
{\bfseries Анонс:}\\\\
«Линия c калибровкой». Особенности движения по линии с двумя датчиками. Соревнования «Линия-2».\\\\
{\bfseries Цели:}
\begin{itemize}
	\item{}{\bfseries Обучающие:} Закрепить знания учащихся о роли и принципах калибровки. Реализовать движение по линии с двумя датчиками освещенности.
	\item{}{\bfseries Развивающая:} Обеспечить развитие навыков анализа своей деятельности и творческого мышления.\\
\end{itemize}	
{\bfseries Ход занятия:}\\\\
\begin{tabular}[h!]{lll}
	{\hyperlink{lesson17x1}{1. Организационный момент}}&{Презентация}&{(5 мин)}\\
	{\hyperlink{lesson17x2}{2. Линия с калибровкой}}&{Игра}&{(40 мин)}\\
	{\hyperlink{lesson17x3}{3. Анализ технических решений}}&{Рефлексия}&{(10 мин)}\\
	{\hyperlink{lesson17x4}{4. «Линия-2»}}&{Презентация}&{(10 мин)}\\
\end{tabular}\\\\

{\hypertarget{lesson17x1}{\blackBlueText{I. Организационный момент}}}\\\\	

В ходе данного занятия командам потребуются полигоны Линия и , возможно, Линия–профи. Желательно иметь два полигона каждого вида: для тренировок команд и для соревновательных заездов на время. Кроме этого каждой команде понадобится рабочее место с набором и компьютером.\\\\
\clearpage
{\hypertarget{lesson17x2}{\blackBlueText{II. Линия с калибровкой}}}\\\\

{\slshape Следующее соревнование практически повторяет соревнования, проведенные на прошлом занятии, однако включают в себя обязательную калибровку робота перед стартом. Способ калибровки не регламентирован. Соревнования лучше сделать командными, команда~--- 2--3 человека.}\\\\
Регламент соревнований:

\begin{enumerate}
	\item За наиболее короткое время робот, следуя черной линии, должен добраться от места старта до места финиша.
	\item  {\bfseries Если робот потеряет линию более чем на 5 секунд, он будет дисквалифицирован.} (Покидание линии, при котором никакая часть робота не находится над линией, может быть допустимо только по касательной и не должно быть больше чем три длины корпуса робота. Длина робота в этом случае считается по колесной базе.)
	\item Перед началом движения робот должен произвести  калибровку: считать значения черного и белого цвета и использовать их в дальнейшей программе.
	\item Жюри вправе запросить текст программы робота. 
	\item Во время проведения состязания участники команд не должны касаться роботов.
	\item Команде дается не менее двух попыток. В зачет идет лучшее время. 
	\item Победителем будет объявлена команда, потратившая на преодоление дистанции наименьшее время.\\\\
\end{enumerate}

{\hypertarget{lesson17x3}{\blackBlueText{III. Анализ технических решений}}}\\\\

\begin{enumerate}
	\item Улучшало ли наличие калибровки результат вашего робота?
	\item Какой метод калибровки использовала команда?
	\item Какой, из использованных всеми участниками, метод калибровки, на ваш взгляд, оптимален? Почему?
	
	
	{\slshape Существует несколько стандартных методов которые стоит обсудить. Можно предложить роботу считать показания датчика освещенности по нажатию кнопки и запомнить их как черный и белый. Можно запоминать показания по маху рукой перед сонаром. Можно поставить робота изначально «глазом» над черным, он  запомнит показания, повернется, окажется над белым, запомнит показания, повернется обратно.}
	\item Изменяли ли вы коэффициент усиления при увеличении скорости движения робота? В какую сторону? Почему?
	\item Пришлось ли вносить изменения в конструкцию робота в связи с увеличением его скорости?\\\\
\end{enumerate}

{\hypertarget{lesson17x4}{\blackBlueText{IV. «Линия-2»}}}\\\\

Вспомним кратко алгоритм движения по линии с одним датчиком. Датчик находится с одной стороны линии, алгоритм основан на пропорциональном регуляторе, где величина воздействия определяется коэффициентом усиления и разностью текущего показания датчика и фиксированной величины.

\begin{equation}
up=k*(S1-param)
\end{equation}	

Основные недостатки алгоритма с одним датчиком~--- обязательная калибровка, сильная чувствительность к освещенности, чувствительность к толщине линии (плохо работает на тонких линиях при поворотах в сторону, с которой установлен датчик), работает только на сплошной линии. Кроме этого, в общем случае линия имеет повороты как налево, так и направо. В то же время, датчик у робота установлен с одной выбранной стороны. Подобное отсутствие симметрии у робота при наличии симметрии у полигона наводит на соображения о том, что такой подход не самый оптимальный.

Попробуем устранить несимметричность и проанализировать результаты. Установим два датчика освещенности с двух сторон от линии. Величина воздействия будет теперь определяться по формуле:

\begin{equation}
up=up_1-up_2=k1*(S1-param)-k2*(S2-param)
\end{equation}	

Где \(up_1\) – величина воздействия от первого датчика, \(up_2\)~--- от второго. Так как датчики установлены с разной стороны от линии, то знаки у \(up_1\) и \(up_2\) разные. Так как левый датчик работает в таких же условиях, что и правый, логично предположить, что коэффициенты усиления у них должны быть одинаковы. Учитывая это, раскроем скобки:

\begin{equation}
up=k*S1-k*param-k*S1+k*param=k*(S1-S2)
\end{equation}	

Проанализируем получившийся результат. Мы получили, что величина воздействия зависит только от коэффициента усиления и разности показаний датчиков. Разберемся, что это значит. Если левый датчик регистрирует более «темные» показания чем правый, то вероятнее всего левый датчик находится над линией, а правый датчик вне линии. Поворачивать в таком случае необходимо налево. Так как значение «серого» явно не входит в выражение для величины воздействия, то в первом приближении калибровка не требуется. 

Рассмотрим некоторые тонкости такого алгоритма и его преимущества, особенности и недостатки. 

Алгоритм с двумя датчиками при всех прочих равных условиях может работать на линиях примерно в два раза тоньше, чем это необходимо для алгоритма с одним датчиком. Дело в том, что алгоритм с одним датчиком выходит из зоны работы при пересечении центром датчика середины линии (показания освещенности уменьшаются и алгоритм поворачивает тележку не в ту сторону). Алгоритм с двумя датчиками «понимает» положение тележки относительно линии вплоть до того момента, как датчик видит хоть какую-то часть линии, то есть показания одного датчика хоть немного «темнее» показаний другого. Описанное выше показано на рисунке:\\\\

\greenText{Рисунок показывающий, что алгоритму с двумя датчиками достаточно долее тонкой линии}\\\\

Еще одно преимущество алгоритма с двумя датчиками~--- он может работать не только на сплошных линиях, но и на пунктирных, из точек, елочек и т.п. Такие преимущества по сравнению с алгоритмом с одним датчиком появляются из-за того, что тележка продолжает двигаться прямо, не сворачивая, если оба датчика находятся на белом фоне. Алгоритм с одним датчиком, как только линия прервется, начнет поворачивать тележку, и с большой вероятностью линия будет потеряна. Алгоритм с двумя датчиками продолжит в такой ситуации двигаться прямо, пока не встретит очередной участок и не сориентируется по нему (при правильно подобранном коэффициенте усиления).

Опыт показывает, что тщательно настроенная система при попадании в другие условия освещенности все же порой начинает работать менее стабильно, сбивается. Разберемся, почему так происходит. Напомним, что показания датчика освещенности – цифровые, то есть, образно выражаясь, датчик сравнивает текущие показания освещенности с таблицей уровней освещенности (абсолютных, аппаратным образом) и выдает номер уровня. \greenText{(см. рисунок)}\\\\

\greenText{Иллюстрация регистрации уровня освещенности цифровым датчиком}\\\\

Номер уровня это и есть то число, которое возвращает команда SensorValue. Так как таблица уровней абсолютная, то в темном помещении количество различимых уровней будет меньше, чем в светлом помещении.  Освещенность в комнате будет включать меньшее количество уровней освещенности датчика (см. \greenText{рисунок}).\\\\

\greenText{Иллюстрация: в темном помещении количество различимых уровней освещенности меньше, чем в светлом}\\\\

В зависимости от количества разрешаемых уровней освещенности будет меняться плавность работы пропорционального регулятора. Количество возможных значений величины воздействия равно количеству разрешаемых уровней освещенности (см формулу). То есть, если датчик разрешает только 2 уровня, то пропорциональный регулятор превращается в релейный, то есть довольно нестабильный. Кроме этого, в зависимости от источника света, черная линия может иметь немного разное освещение. А также иногда встречаются глянцевые поля, механизм отражения света в которых близок к зеркальному.

Именно об этих трех факторах идет речь, когда говорят о том, что коэффициент усиления зависит от освещенности. В итоге, алгоритм движения с двумя датчиками существенно меньше зависит от уровня освещенности в помещении, однако процедура калибровки может оказаться необходимой, поэтому автор рекомендует по-прежнему ее использовать.

В заключение разговора о движении по линии с двумя датчиками учащимся предлагается реализовать этот алгоритм на практике.

{\slshape При наличии времени, возможно проведение соревнований по регламенту, аналогичному пункту II, но для роботов с двумя датчиками.
Для сильных групп можно предложить сразу провести подобные соревнования на пунктирной линии (план поля в Приложении).

Для очень сильных групп можно посветить еще одно занятие следующей задаче.

\underline{Дополнительно.} Для очень сильных групп можно посветить еще одно занятие следующей задаче.Учащимся предлагается изучить зависимость коэффициента усиления от расстояния между линией крепежа датчиков и осью ведущих колес. Вначале необходим теоретический анализ (можно провести дома), после этого теоретические результаты необходимо проверить на практике.}

\textcolor[rgb]{1,0,0}{Коэффициент усиления зависит от очень многих параметров, таких как ширина линии, уровень освещенности в помещении, геометрических размеров робота и положения датчика, веса робота, трения и поверхность, трения в осях и зубчатых передачах, заряда батареи и других. Поэтому не представляется возможным получить универсальное выражение, позволяющее рассчитать коэффициент усиления для конкретного случая. Можно лишь получить зависимости от перечисленных параметров, которые помогут подобрать правильный коэффициент.}